\documentclass[a4paper,12pt]{article}
\usepackage[dvips]{graphicx,epsfig}
\usepackage{amsmath}
\usepackage{amsthm}
\usepackage{amssymb}
\usepackage[boxed]{algorithm2e}
\usepackage{color}
 
\newtheorem{thm}{Theorem}[section]
\newtheorem{cor}[thm]{Corollary}
\newtheorem{lem}[thm]{Lemma}
\newtheorem{defn}[thm]{Definition}


%\newcommand{\argmax}{\arg\!\max}
\newcommand{\argmax}{\operatornamewithlimits{argmax}}

\makeatletter
\renewcommand{\env@cases}[1][@{}l@{\quad}l@{}]{%
  \let\@ifnextchar\new@ifnextchar
  \left\lbrace
    \def\arraystretch{1.2}%
    \array{#1}%
}
\makeatother

\newcommand\hcancel[2][black]{\setbox0=\hbox{$#2$}%
\rlap{\raisebox{.45\ht0}{\textcolor{#1}{\rule{\wd0}{1pt}}}}#2} 

\begin{document}

\section{Mitonchondrial}

Let $r$ be a read of length $l$ with sequenced bases = $r_1 r_2 ... r_l$ with error probabilities  $\epsilon_1 \epsilon_1 ... \epsilon_l$.  Let the read $r$ have the probability of mismapping $m$. 

\subsection{Aligned bases}

\noindent Let the base $b\in\{A,C,G,T\}$, the probability of observing the base $r_i$ from read $r$, given that it was properly aligned and given that $b$ is the correct base, is computed by:

\begin{equation}
  p_{mapped}(r_i|b)   = \begin{cases}[@{}l@{\quad}r@{}l@{}]
    1-\epsilon_i  &  \text{if }  b = r_i    \\
    \frac{ \epsilon_1} {3} &  \text{if }  b \ne r_i    \\
  \end{cases}
\end{equation} 

\noindent However, if the read was mismapped we revert to our prior:


\begin{equation}
  p_{mismapped}(r_i|b)   =  p(b) =     \frac{ 1} {4} 
\end{equation} 

\noindent Finally, we can combine both:

\begin{equation}
  p(r_i|b)   =  (1-m) \cdot p_{mapped}(r_i|b) + m \cdot p_{mismapped}(r_i|b) 
\end{equation} 


\noindent To compute the likelihood of the base $b$, 
\begin{equation}
p(b|r) = \frac {p(r|b) \cdot p(b)} {p(r)}
\end{equation} 

Assuming a uniform prior  $p(b) = \frac{ 1} {4}$ and that $p(r)$ can be obtained using a marginalization over each base $b\in\{A,C,G,T\}$, the likelihood of the base $b$ given $r$ is proportional to probability of generating $r$ given $b$ :

\begin{equation}
p(b|r) \propto p(r|b)
\end{equation} 

Given multiple reads such that $r \in R$, we assume that each read is an independent observation:

\begin{equation}
p(b|R) = \prod_{r \in R} p(b|r)
\end{equation} 

Finally we retain $b'$ the most likely base such that 

\begin{equation}
 b' = \argmax_{b \in \{A,C,G,T\} }   p(b|R)
\end{equation} 


The probability of error on $b'$ is given by:

\begin{equation}
p(\neg b|R) = \frac { \sum_{ b \in \{A,C,G,T\}  \setminus b' } p(b|R) } { \sum_{ b \in \{A,C,G,T\}  } p(b|R) }
\label{errormt}
\end{equation}
 



\subsection{Indels}


We use majority vote :-(


%\begin{equation}
%  = (1-\epsilon_i) \frac{ \epsilon_1} {3}
%\end{equation}


\section{Mitonchodrial contamination}

Let $b$ be the base from the endogenous sample and $c$ be the base from the contaminant. Let the contamination rate be $c_r$, defined as the probability of a base from the contaminant at any given position. Therefore, the probability that the allele is endogenous, is $1-c_r$.

\begin{equation}
p(R|b,c)  = (1-m) \cdot p_{mapped}(r_i|b,c) + m \cdot p_{mismapped}(r_i|b,c)  %\sum_{ b,c \in \{AC,AG,AT,CA,CG,CT,GA,GC,GT,TA,TC,TG\} } p(R|b,c) p(b,c)
\end{equation}

\begin{equation}
p_{mapped}(r_i|b,c) = \sum_{ b,c \in \{AC,AG,AT,CA,CG,CT,GA,GC,GT,TA,TC,TG\} } p(r_i|b,c) p(b,c)
\end{equation}

\begin{equation}
p(b,c) = p(b) \cdot p(c)
\end{equation}

From \ref{errormt}, we get:

\begin{equation}
p(b)  = 1 - p(\neg b|R)
\end{equation}

We get $p(c)$ using the allele frequency in human populations.



\begin{equation}
p(r_i|b,c)  =  p(r_i|b) \cdot p(b)  +  p(r_i|c) \cdot p(c) 
\end{equation}


\begin{equation}
p(r_i|b,c)  =  [1-c_r] p(r_i|b) +  [c_r] p(r_i|c)
\end{equation}




\clearpage
\section{Deamination}


\begin{equation}
p(R|b,c)  = (1-m) \cdot p_{mapped}(r_i|b,c) + m \cdot p_{mismapped}(r_i|b,c)  %\sum_{ b,c \in \{AC,AG,AT,CA,CG,CT,GA,GC,GT,TA,TC,TG\} } p(R|b,c) p(b,c)
\end{equation}


\section{Nuclear Contamination}

\begin{figure}
  % 
  %   \includegraphics[height=8cm]{../merging_crop}
  % \end{center}
  %\begin{tabular}{c}
  \begin{center}
    \includegraphics{figures1_cropped_now}  
  \end{center}
   % \end{tabular}
  \caption{}\label{fig1}
\end{figure}


\begin{equation}
p(\mathcal{M} | \mathcal{D}   ) \propto  p(\mathcal{D}  | \mathcal{M} ) \cdot p(\mathcal{M} )
\end{equation}
The model $\mathcal{M}$, contains terms for :
\begin{itemize}
\item heterozygosity
\item  contamination
\item  deamination
\item sequencing error
\end{itemize}

$\mathcal{M} = {h,e,c}$
where:
\begin{eqnarray*}
h & = & heterozygosity\ rate \\
e & = & Illumina\ error\ rate \\
c & = & contamination\ rate \\
\end{eqnarray*}

Goal : Plot the joint post. prob for ${h,c}$ and post. prob for $e$.

\begin{equation}
p(\mathcal{M} | \mathcal{D}   ) \propto \sum_{m \in \{ h,c,d,e\} }  p(\mathcal{D}  | \mathcal{M}_{m} ) \cdot p(\mathcal{M}_{m} )
\end{equation}




\subsection{Illumina Error}

Let $b$ be the most likely base under a homozygous model, let $b'$ be the $k$ minority bases. ($n-k$ bases that are $b$). 

\begin{equation}
p(\mathcal{D}  | \mathcal{M}_{e} ) \cdot p(\mathcal{M}_{e} ) = [ \prod_{i=1}^k \epsilon_i \cdot prob_{error}(b,b') \prod_{i=k+1}^n ({1-\epsilon_i}) ] \cdot e
\end{equation}

\noindent Simple model: all substitution equally likely:

\begin{equation}
prob_{error}(b,b')   = \frac {1} {3}
\end{equation}

\noindent Second: use estimated nucleotide conversion rates:
\begin{equation}
prob_{error}(b,b')   = \frac {\# b \rightarrow b'} {\sum\limits_{n\in\{A,C,G,T\} \setminus b}{\# b \rightarrow n}}
\end{equation}




\subsection{Contamination}

Let $b$ be the most likely base under a homozygous model, let $b'$ be the $k$ minority bases. ($n-k$ bases that are $b$). 

\begin{equation}
p(\mathcal{D}  | \mathcal{M}_{c} ) \cdot p(\mathcal{M}_{c} ) = 
c \cdot f \cdot \prod_{i=1}^k (1-\epsilon_i)  + (1-c) \cdot \prod_{i=k+1}^n ({1-\epsilon_i}) 
\end{equation}


\subsection{Deamination}
Let $b$ be the most likely base under a homozygous model, let $b'$ be the $k$ minority bases. ($n-k$ bases that are $b$). 

\begin{equation}
p(\mathcal{D}  | \mathcal{M}_{d} ) \cdot p(\mathcal{M}_{d} ) =  \prod_{i=1}^k (1-\epsilon_i) \cdot p(D|b_i,p_i )  + \prod_{i=k+1}^n ({1-\epsilon_i}) \cdot p(D|b_i,p_i )
\end{equation}


\subsection{Heterozygosity}

Let $\mathcal{D}$ be the observed data and $\mathcal{G}$ the genotype. 

\subsection{$n=1$}

\begin{eqnarray*}
p(\mathcal{D} = \langle b \rangle | \mathcal{G} = \langle b,b' \rangle) & = & p( b | r_1 = b   )   \cdot p( r_1 =  b) \\
                                                                        &   & p( b | r_1 = b'   )   \cdot p( r_1 =  b')  \\
\end{eqnarray*}

\begin{eqnarray*}
p(\mathcal{D} = \langle b \rangle | \mathcal{G} = \langle b,b' \rangle) &  = & ( 1-\epsilon_1)   \cdot \frac {1} {2} \\
                                                                        &    & \hcancel{  (\frac{ \epsilon_1} {3} ) \cdot \frac {1} {2}} \\
\end{eqnarray*}

\subsection{$n=2$}
\begin{eqnarray*}
p(\mathcal{D} = \langle bb' \rangle| \mathcal{G} = \langle b,b' \rangle) & = &  p( bb' | r_1 = b  \land r_2 = b  )   \cdot p( r_1 =  b \land r_2 =  b)  + \\
                             &   &  p( bb' | r_1 = b  \land r_2 = b' )   \cdot p( r_1 =  b \land r_2 =  b')  +     \\
                             &   &  p( bb' | r_1 = b' \land r_2 = b  )   \cdot p( r_1 =  b' \land r_2 =  b)  +     \\
                             &   &  p( bb' | r_1 = b' \land r_2 = b' )   \cdot p( r_1 =  b' \land r_2 =  b')       \\                             
\end{eqnarray*}


\begin{eqnarray*}
p(\mathcal{D} = \langle bb' \rangle| \mathcal{G} = \langle b,b' \rangle) & = & \hcancel{ ( 1-\epsilon_1) (\frac{ \epsilon_2} {3})     \cdot \frac {1} {4} }\\            
                                                                         &   &  ( 1-\epsilon_1)( 1-\epsilon_2)    \cdot  \frac {1} {4} \\            
                                                                         &   &  ( 1-\epsilon_1)( 1-\epsilon_2)   \cdot  \frac {1} {4} \\            
                                                                         &   & \hcancel{ (\frac{ \epsilon_1} {3}) ( 1-\epsilon_2)       \cdot  \frac {1} {4}} \\            
\end{eqnarray*}

\subsection{$n=3$}

\begin{eqnarray*}
p(\mathcal{D} = \langle bbb' \rangle| \mathcal{G} = \langle b,b' \rangle) & = &  \hcancel{p( bbb' | r_1 = b  \land r_2 = b \land r_3 = b ) \cdot p( r_1 = b  \land r_2 = b \land r_3 = b  ) } \\
                                                                          &   &  p( bbb' | r_1 = b  \land r_2 = b \land r_3 = b' ) \cdot p( r_1 = b  \land r_2 = b \land r_3 = b'  ) \\
                                                                          &   &  p( bbb' | r_1 = b  \land r_2 = b' \land r_3 = b ) \cdot p( r_1 = b  \land r_2 = b' \land r_3 = b  ) \\
                                                                          &   &  \hcancel{p( bbb' | r_1 = b  \land r_2 = b' \land r_3 = b' ) \cdot p( r_1 = b  \land r_2 = b' \land r_3 = b'  ) }\\
%                                                                         
                                                                          &   &  p( bbb' | r_1 = b'  \land r_2 = b \land r_3 = b ) \cdot p( r_1 = b'  \land r_2 = b \land r_3 = b  ) \\
                                                                          &   &  \hcancel{p( bbb' | r_1 = b'  \land r_2 = b \land r_3 = b' ) \cdot p( r_1 = b'  \land r_2 = b \land r_3 = b'  )} \\
                                                                          &   &  \hcancel{p( bbb' | r_1 = b'  \land r_2 = b' \land r_3 = b ) \cdot p( r_1 = b'  \land r_2 = b' \land r_3 = b  )} \\
                                                                          &   &  \hcancel{p( bbb' | r_1 = b'  \land r_2 = b' \land r_3 = b' ) \cdot p( r_1 = b'  \land r_2 = b' \land r_3 = b'  ) } \\
\end{eqnarray*}

\subsection{$n=k$}
\begin{equation}
p(\mathcal{D} = \langle bxk , b'x(n-k) \rangle| \mathcal{G} = \langle b,b' \rangle) = {n \choose k} \cdot \frac {1} {2^n} \cdot \prod_{i=1}^n (1-\epsilon_{i})
\end{equation}


\subsection{Questions:}


\begin{itemize}
\item Divergence between sample and reference 
\item Mapping quality as a prob ?
\item Multithreading + Markov chain
\end{itemize}


%
%
%\begin{eqnarray*}
%p(\mathcal{G} = \langle \rangle | \mathcal{G} = \langle b,b' \rangle) & = & \frac {1} {2} (p( first\ sampled\ b) )  + \frac {1} {2} (p( first\ sampled\ b)' ) \\
%                           & = & \frac {1} {2} (1-\epsilon_1)            + \frac {1} {2} (\frac{ \epsilon_1} {3} ) \\
%\end{eqnarray*}
%
%
%\begin{eqnarray*}
%p(bb| \langle b,b' \rangle) & = & \frac {1} {4} p( first\ sampled\ b)       \cdot p(second\ sampled\ b)   + \\
%                            &   & \frac {1} {4} p( first\ sampled\ b)       \cdot p(second\ sampled\ b')  + \\
%                            &   & \frac {1} {4} p( first\ sampled\ b')      \cdot p(second\ sampled\ b)   + \\
%                            &   & \frac {1} {4} p( first\ sampled\ b')      \cdot p(second\ sampled\ b')    \\
%%%%%
%                            & = & \frac {1} {4} ( 1-\epsilon_1)            \cdot (1-\epsilon_2)  + \\
%                            &   & \frac {1} {4} ( 1-\epsilon_1)            \cdot (\frac{\epsilon_2} {3})  + \\
%                            &   & \frac {1} {4} ( \frac{\epsilon_1} {3})   \cdot (1-\epsilon_2)   + \\
%                            &   & \frac {1} {4} ( \frac{\epsilon_1} {3})   \cdot (\frac{\epsilon_2} {3})  \\
%\end{eqnarray*}
%
%\begin{eqnarray*}
%p(bb'| \langle b,b' \rangle) & = & \frac {1} {4} p( first\ sampled\ b)       \cdot p(second\ sampled\ b)  + \\
%                             &   & \frac {1} {4} p( first\ sampled\ b)       \cdot p(second\ sampled\ b')  + \\
%                             &   & \frac {1} {4} p( first\ sampled\ b')      \cdot p(second\ sampled\ b)   + \\
%                             &   & \frac {1} {4} p( first\ sampled\ b')      \cdot p(second\ sampled\ b') \\
%%%%%
%                            & = & \frac {1} {4} (   (1-\epsilon_1 ) \cdot (\frac{\epsilon_2} {3})    + (\frac{\epsilon_1} {3} ) \cdot (1-\epsilon_2) ) \\
%                            &   & \frac {1} {4} ( 1-\epsilon_1)            \cdot (1-\epsilon_2)  + \\
%                            &   & \frac {1} {4} ( 1-\epsilon_1)            \cdot (1-\epsilon_2)   + \\
%                            &   & \frac {1} {4} (   (1-\epsilon_1 ) \cdot (\frac{\epsilon_2} {3})  + (\frac{\epsilon_1} {3} ) \cdot (1-\epsilon_2) ) \\
%\end{eqnarray*}
%or can be written as:
%
%\begin{eqnarray*}
%p(bbb| \langle b,b' \rangle) & = & \frac {1} {8} p( first\ sampled\ b)       \cdot p(second\ sampled\ b)     \cdot p(third\ sampled\ b)   + \\
%                             &   & \frac {1} {8} p( first\ sampled\ b)       \cdot p(second\ sampled\ b)     \cdot p(third\ sampled\ b')  + \\
%                             &   & \frac {1} {8} p( first\ sampled\ b)       \cdot p(second\ sampled\ b')    \cdot p(third\ sampled\ b)  + \\
%                             &   & \frac {1} {8} p( first\ sampled\ b)       \cdot p(second\ sampled\ b')    \cdot p(third\ sampled\ b')  + \\
%%%%
%                             &   & \frac {1} {8} p( first\ sampled\ b')       \cdot p(second\ sampled\ b)     \cdot p(third\ sampled\ b)   + \\
%                             &   & \frac {1} {8} p( first\ sampled\ b')       \cdot p(second\ sampled\ b)     \cdot p(third\ sampled\ b')  + \\
%                             &   & \frac {1} {8} p( first\ sampled\ b')       \cdot p(second\ sampled\ b')    \cdot p(third\ sampled\ b)  + \\
%                             &   & \frac {1} {8} p( first\ sampled\ b')       \cdot p(second\ sampled\ b')    \cdot p(third\ sampled\ b')  + \\
%\end{eqnarray*}
%
%
%\begin{eqnarray*} 
%p(bbb| \langle b,b' \rangle) & = & \frac {1} {8} (1-\epsilon_1 )       \cdot (1-\epsilon_2)     \cdot (1-\epsilon_3)   + \\
%                             &   & \frac {1} {8} (1-\epsilon_1 )       \cdot (1-\epsilon_2)     \cdot (\frac{\epsilon_3} {3} )  + \\
%                             &   & \frac {1} {8} (1-\epsilon_1 )       \cdot (\frac{\epsilon_2} {3})    \cdot (1-\epsilon_3)  + \\
%                             &   & \frac {1} {8} (1-\epsilon_1 )       \cdot (\frac{\epsilon_2} {3})    \cdot (\frac{\epsilon_3} {3} )  + \\
%%%%
%                             &   & \frac {1} {8} ( \frac{\epsilon_1} {3})       \cdot (1-\epsilon_2)     \cdot (1-\epsilon_3)   + \\
%                             &   & \frac {1} {8} ( \frac{\epsilon_1} {3})       \cdot (1-\epsilon_2)     \cdot (\frac{\epsilon_3} {3} )  + \\
%                             &   & \frac {1} {8} ( \frac{\epsilon_1} {3})       \cdot (\frac{\epsilon_2} {3})   \cdot (1-\epsilon_3)  + \\
%                             &   & \frac {1} {8} ( \frac{\epsilon_1} {3})       \cdot (\frac{\epsilon_2} {3})    \cdot (\frac{\epsilon_3} {3} ) + \\
%\end{eqnarray*}
%
%\begin{eqnarray*}
%p(bbb'| \langle b,b' \rangle) & = & \frac {1} {8} p( first\ sampled\ b)       \cdot p(second\ sampled\ b)     \cdot p(third\ sampled\ b)   + \\
%                              &   & \frac {1} {8} p( first\ sampled\ b)       \cdot p(second\ sampled\ b)     \cdot p(third\ sampled\ b')  + \\
%                              &   & \frac {1} {8} p( first\ sampled\ b)       \cdot p(second\ sampled\ b')    \cdot p(third\ sampled\ b)  + \\
%                              &   & \frac {1} {8} p( first\ sampled\ b)       \cdot p(second\ sampled\ b')    \cdot p(third\ sampled\ b')  + \\
%%%%
%                              &   & \frac {1} {8} p( first\ sampled\ b')       \cdot p(second\ sampled\ b)     \cdot p(third\ sampled\ b)   + \\
%                              &   & \frac {1} {8} p( first\ sampled\ b')       \cdot p(second\ sampled\ b)     \cdot p(third\ sampled\ b')  + \\
%                              &   & \frac {1} {8} p( first\ sampled\ b')       \cdot p(second\ sampled\ b')    \cdot p(third\ sampled\ b)  + \\
%                              &   & \frac {1} {8} p( first\ sampled\ b')       \cdot p(second\ sampled\ b')    \cdot p(third\ sampled\ b')  + \\
%\end{eqnarray*}
%
%
%\begin{eqnarray*}
%p(bbb'| \langle b,b' \rangle) & = & \frac {1} {8} p( first\ sampled\ b)       \cdot p(second\ sampled\ b)     \cdot p(third\ sampled\ b)   + \\
%                              &   & \frac {1} {8} p( first\ sampled\ b)       \cdot p(second\ sampled\ b)     \cdot p(third\ sampled\ b')  + \\
%                              &   & \frac {1} {8} p( first\ sampled\ b)       \cdot p(second\ sampled\ b')    \cdot p(third\ sampled\ b)   + \\
%                              &   & \frac {1} {8} p( first\ sampled\ b)       \cdot p(second\ sampled\ b')    \cdot p(third\ sampled\ b')  + \\
%%%%
%                              &   & \frac {1} {8} p( first\ sampled\ b')       \cdot p(second\ sampled\ b)     \cdot p(third\ sampled\ b)   + \\
%                              &   & \frac {1} {8} p( first\ sampled\ b')       \cdot p(second\ sampled\ b)     \cdot p(third\ sampled\ b')  + \\
%                              &   & \frac {1} {8} p( first\ sampled\ b')       \cdot p(second\ sampled\ b')    \cdot p(third\ sampled\ b)   + \\
%                              &   & \frac {1} {8} p( first\ sampled\ b')       \cdot p(second\ sampled\ b')    \cdot p(third\ sampled\ b')  + \\
%\end{eqnarray*}
%
%
%
%
%
\end{document}
%yyy
